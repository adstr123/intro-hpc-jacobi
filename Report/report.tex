\documentclass[10pt]{article}
\usepackage{graphicx}

\begin{document}
\title{Optimising the Serial Jacobi Code}
\author{Adam Matheson \\
Student Number: P 1757290}
\date{\today}
\maketitle

\section{Introduction}
As parallelization continues to increase in the world of
high-performance computing, optimising the serial code beforehand is
more important than ever. Defaulting to parallelizing one's code may
seem like a tempting choice when it has the potential to offer such
large performance increases without the tedium of profiling and
analyzing code in detail as required with optimisation, but it would
be remiss to leave such potential performance on the table.

To demonstrate this, what follows is a walkthrough of a successful
attempt to optimise serial code running the Jacobi algorithm for
solving a set of linear equations.

\section{Optimisations}
\subsection{Baseline}
In order to make meaningful assessments of the performance gains made
whilst optimising, baseline performance is required beforehand. From
here on unless stated otherwise, performance will be tested on matrix
orders 1000, 2000, and 4000 to give a range of figures, so places
where Xs/Yz/Zs is seen refers to run time for matrix order 1000 = X
(in seconds), 2000 = Y (in seconds), and 4000 = Z (in seconds). Run
times are averaged and rounded.

Running the code as-is on BlueCrystal Phase 3, compiled with GNU
compiler without any optimisation flags or debugging or any
improvements to the stock code, gave times of 10.91/130.75/1180.81s.

\subsection{Compiler}
The first crucial step to optimisation is of course
profiling. Learning where the slow part of your code is is essential
for efficient use of optimisation time; even eliminating 99.999\% of
the run time of a code block that only takes up 1\% of the overall
program run time is still only eliminating ~1\% of the program run
time. By contrast, reducing a code block that comprises 90\% of the
total run time by 10\% eliminates 9\% of the total run time.

However, before beginning profiling, I switched compilers from GNU to
Intel C Compiler (icc). Whilst the GNU compiler may indeed be better
for some things, I didn't need profiling to know the Intel compiler
offers better optimisations than GNU. Besides being the general
concensus, in speed tests such as \textbf{X AND Y} prove it compiles
to faster executables. Indeed, after this change, the timings were
3.27s/53.35s, nearly 3 times as fast without changing a single line of
code myself.

\subsection{First profile and offending code identification}
Now the preliminary steps had been taken care of it was time to
profile. After a short comparison of profiling tools including gperf,
tau, valgrind, gperftools, and Intel VTune, I settled on
gperftools. It struck a balance between detail and complexity. Gprof
seemed too simple from initial tests, although it is installed by
default on Linux systems. At the other end of the scale VTune seemed
to offer incredible flexibility, but was more difficult to manipulate
in order to get the exact desired output.

Whilst gperftools was not installed on BlueCrystal, I could profile
locally since the processing power required was not high and I didn't
need like-for-like run times; I just wanted to identify which code
blocks were taking up \emph{percentages} of run time. After installing
libunwind, dot, dv and EPEL, gperftools could be used. This simply
involved including the header file and invoking
``ProfilerStart(<outputfile>)'' and ``ProfilerStop()'' functions, then
recompiling with -g debug flag and linking to the profiler with
-lmprofiler. It is worth mentioning that debugging increased the run
time significantly to around 10 seconds for the 1000 matrix, but once
again, I am simply trying to identify percentage of run times. The
profile is accessed with pprof --text --lines ./<program>
<profilefile>.

Immediately gperftools identified line 64 of the default code as an
offender occupying ~85\% of the total run time. I identified this as a
loop in the run() function where the Jacobi algorithm is executed, and
the dot product of the inputs and the x values is calculated.

\subsection{Investigation and cachegrind cache analysis}
Which line of code was the offending line may have identified by the
profiling tool but it was still up to human logic to figure out
exactly why this line was offending. To figure this out, I relied on
the trusty pen and paper - empirically the best debugger around
(reference). I reduced the matrix order to 3 and drew out step by step
which element of the array is being checked and assigned for each
value of row and col (9 values in total). When I visually represented
the single dimension array as a 2D array on paper it became clear the
data was being accessed column-wise. Since in C arrays are stored
row-wise, this means that the memory was being accessed in an
inefficient manner - jumping around incontiguously - which involves
memory being loaded in and out throughout the memory heirarchy and
wasting CPU cycles.

To confirm this suspicion, I installed valgrind and utilised it's
--cachegrind option which analyses how memory is being accessed, from
instruction registers through L1 and L2 caches. The following results
were produced:

\includegraphics[scale=0.5]{../cachegrind}

This was actually a little better than I anticipated but 7\% is still
a very high percentage and would represent a large performance issue.

\subsection{Manipulating the data structure}
I had to make the algorithm access the array elements sequentially; so
the ``rows'' of the 2D array had to be where the ``columns'' currently
were. Altering the data initialisation by ``transposing'' the array
using different indexing achieved this (proved with a debug loop to
print the array before and after the code block). This resulted in a
larger overall time due to the transposition but a lower solver run
time, proving the concept. I then worked this code into the main
initialisation block.

To match this, the Jacobi solver was also modified to index
differently. This sent the time plummeting to 1.1s/y/90s.

\subsection{Additional optimisations}
Working through a structured checklist of other optimisations.

\subsubsection{Precision}
I decided that the precision was too high. I changed doubles to floats
which resulted in an extremely small performance increase of ~5\%,
which could be attributed to random changes. Timings after this were
1.04s/8.9s/84.7s.

\subsubsection{Compiler Flags}
As compiler optimisation through flags can have mixed and seemingly
arbitrary effects, I decided to leave these til the end. Enabling -O3
aggressive optimisation gave times of 1.01s/7.1s/73.3s.

Enabling -no-prec-div gave times of 1.02/7.23s/73.37s, giving no
benefit.

Enabling -Ofast gave 1.02s/7.18s/77.32s.

\subsubsection{Loop Fusion}
Combining loops gave 0.44s/2.11s/13.43s but increased the error to an
inappropriate level. Given more time, loop fusion would have been
worth pursuing.

\subsubsection{Vectorisation}
Enabled -qopt-report and viewed jacobi.optrpt. Vectorisation taking
place mostly, some unable to vectorise.

\section{Conclusion}
Final times were 0.99/6.81/66.61. Overall this exercise proves there
is massive value to be had in optimisation before parallelization. In
a time when high performance computing resources are treated as lab
tools, allocated by the second, every time saving operation
counts. Whilst modern compilers can do a lot of work for you, there is
still a lot to be said for the programmer optimising options and code
herself.

\subsection{Going Further}
Further optimisations
Parallelize using blah blah

\end{document}